%% !TeX root = filename
\documentclass[12pt,letterpaper,reqno,oneside]{amsart}
\usepackage{standalone}
\usepackage{subfiles}
\usepackage{subcaption}
\usepackage{dcolumn}
\usepackage[margin=1in]{geometry}
\usepackage[utf8]{inputenc}
\usepackage[T1]{fontenc}
% \usepackage[spanish,es-nodecimaldot,safe=none]{babel}
\usepackage{amsmath}
\usepackage{booktabs}
\usepackage{threeparttable} % for table notes
\usepackage[backend=biber,style=authoryear,natbib=true]{biblatex}
\addbibresource[label=main]{references.bib}
\usepackage{mdframed}
\usepackage{amsfonts}
\usepackage{physics}
% \usepackage{cmbright}
\usepackage{calc}
\usepackage{datetime}
\usepackage[style=american]{csquotes}
\usepackage{mathrsfs}
\usepackage{amssymb}
\usepackage{amsthm}
% \usepackage[extreme]{savetrees}
\usepackage{caption}
\usepackage{graphicx}
\usepackage{xfrac}
\usepackage{tikz}
\usepackage{pgfplots}
\usepackage{marginnote}
\usepackage{listings}
\usepackage[shortlabels]{enumitem}
\usepackage[hidelinks]{hyperref}
%\usepackage[useregional=text]{datetime2}
\usepackage{placeins}
\setcounter{tocdepth}{2}
\setcounter{secnumdepth}{2}
\pgfplotsset{compat=1.5}
% \setlength\parindent{0pt}


% Equations numbers are preceded by subsection
% \numberwithin{equation}{subsection}

% \theoremstyle{definition}
\newtheorem{thm}{Theorem}
\newtheorem{defn}{Definition}[subsection]
\newtheorem{prop}{Proposition}
\newtheorem{cor}{Corollary}[prop]
\newtheorem{remark}{Remark}
\newtheorem{lemma}{Lemma}[subsection]
% Define a new style for the 'problem' environment
\newtheoremstyle{problemstyle}
  {0pt} % Space above
  {\topsep} % Space below
  {} % Body font
  {0pt} % Indent amount
  {\bfseries} % Theorem head font
  {.} % Punctuation after theorem head
  {.5em} % Space after theorem head
  {} % Theorem head spec (can be left empty)

\theoremstyle{problemstyle} % Apply the new style for 'problem'
\newtheorem{problem}{Problem}
\newtheorem{appendixproblem}{Problem}
\theoremstyle{definition} % Revert to 'definition' style for subsequent environments
\newtheorem{solution}{Solution}[problem]
\newtheorem{appendixsolution}{Solution}[appendixproblem]
\renewcommand{\theappendixproblem}{\thesection}
\renewcommand{\theappendixsolution}{\theappendixproblem~(\alph{appendixsolution})}
\renewcommand{\thesolution}{\theproblem~(\alph{solution})}

% Section styling
% \renewcommand\thesection{\arabic{section}}
% \renewcommand\thesubsection{\thesection.\Alph{subsection}}
% \renewcommand\thesubsubsection{\thesubsection.\arabic{subsubsection}}

% Operators
\DeclareMathOperator{\ar}{AR}
\DeclareMathOperator{\ma}{MA}
\DeclareMathOperator{\arma}{ARMA}
\DeclareMathOperator{\garch}{GARCH}
\DeclareMathOperator{\plim}{p\!\lim}

% Packages for quarto


\begin{document}
\author[F. I. Tappata]{Felipe. I. Tappata}
% \address{Universidad Torcuato Di Tella, Buenos Aires, Argentina}
% \email{ftappata@mail.utdt.edu}
\newdate{date}{27}{06}{2025} % 2025-05-25
\date{\displaydate{date}}

\title[Final Exam]{Datos de Panel: Examen Final}
\begin{abstract}
  Este documento contiene la solución al examen final de la materia \emph{Datos de Panel} en la Universidad Torcuato Di Tella, primer trimestre de 2025.
  El trabajo consiste en la replicación de ciertas tablas y figuras del trabajo de \textcite{al-sadoonSimpleMethodsConsistent2019}, y un análisis de los resultados obtenidos.
  El código usado fue entregado junto a este documento, y se puede encontrar también en el repositorio \url{https://github.com/felipetappata/datos-final}.
  El software usado es Stata, con un uso auxiliar de Python y Bash para el procesamiento de \emph{output} y ayuda en la ejecución simultánea de simulaciones.
  El archivo \texttt{README.md} en la raíz del repositorio contiene instrucciones detalladas para la replicación de los resultados.
\end{abstract}
\maketitle
% Problem statements are enclosed in boxes, with solutions following.
% At the cost of slight repetition, I have attempted to make the solutions self-contained.
\begin{mdframed}
  \begin{problem}
  \label{prob:1}
  Reproduzca las tablas 1 a 3 del trabajo de Sadoon et al.
  \end{problem}
\end{mdframed}
\begin{solution}
  \begin{table}[htbp]
    \centering
    \caption{Sesgo promedio en el modelo $\ar(1)$ ($T = 7$, $500$~replicaciones)}
    \label{tab:table1}
    \begin{tabular}{@{}ccc*{4}{c}@{}}
\toprule
& & & \multicolumn{2}{c}{No endogenous} & \multicolumn{2}{c}{Endogenous} \\
& & & \multicolumn{2}{c}{selection} & \multicolumn{2}{c}{selection} \\
\cmidrule(lr){4-5} \cmidrule(lr){6-7}
Select. & & & (1) & (2) & (3) & (4) \\
Model & $\rho$ & & AB & SYS & AB & SYS \\
\midrule
\midrule
\multicolumn{7}{c}{$N = 500$} \\
\midrule
A & $.25$ & bias & $-0.00\underline{9}49$ & $\phantom{-}0.00\underline{1}26$ & $-0.00\underline{7}38$ & $-0.00\underline{4}45$ \\
& & s.e. & $\phantom{-}0.0\underline{7}145$ & $\phantom{-}0.0\underline{4}560$ & $\phantom{-}0.0\underline{6}323$ & $\phantom{-}0.0\underline{4}589$ \\
A & $.50$ & bias & $-0.0\underline{2}332$ & $\phantom{-}0.00\underline{1}86$ & $-0.0\underline{1}490$ & $-0.0\underline{1}056$ \\
& & s.e. & $\phantom{-}0.\underline{1}1484$ & $\phantom{-}0.0\underline{5}038$ & $\phantom{-}0.0\underline{9}159$ & $\phantom{-}0.0\underline{5}248$ \\
A & $.75$ & bias & $-0.\underline{1}0806$ & $\phantom{-}0.00\underline{4}92$ & $-0.0\underline{3}982$ & $-0.0\underline{1}373$ \\
& & s.e. & $\phantom{-}0.\underline{2}5894$ & $\phantom{-}0.0\underline{5}637$ & $\phantom{-}0.\underline{1}5301$ & $\phantom{-}0.0\underline{6}324$ \\
\midrule
\multicolumn{7}{c}{$N = 5000$} \\
\midrule
A & $.25$ & bias & $-0.00\underline{1}69$ & $\phantom{-}0.000\underline{3}7$ & $-0.000\underline{9}7$ & $-0.00\underline{5}53$ \\
& & s.e. & $\phantom{-}0.0\underline{2}285$ & $\phantom{-}0.0\underline{1}421$ & $\phantom{-}0.0\underline{2}022$ & $\phantom{-}0.0\underline{1}403$ \\
A & $.50$ & bias & $-0.00\underline{4}28$ & $\phantom{-}0.000\underline{2}8$ & $-0.00\underline{2}43$ & $-0.0\underline{1}274$ \\
& & s.e. & $\phantom{-}0.0\underline{3}698$ & $\phantom{-}0.0\underline{1}601$ & $\phantom{-}0.0\underline{2}954$ & $\phantom{-}0.0\underline{1}623$ \\
A & $.75$ & bias & $-0.0\underline{1}489$ & $\phantom{-}5.70 \times 10^{-8}$ & $-0.00\underline{5}73$ & $-0.0\underline{2}187$ \\
& & s.e. & $\phantom{-}0.0\underline{8}199$ & $\phantom{-}0.0\underline{1}832$ & $\phantom{-}0.0\underline{4}830$ & $\phantom{-}0.0\underline{2}084$ \\
\midrule
\multicolumn{7}{c}{$N = 500$} \\
\midrule
B & $.25$ & bias & $-0.0\underline{1}137$ & $\phantom{-}0.00\underline{3}30$ & $-0.00\underline{6}97$ & $-0.00\underline{7}21$ \\
& & s.e. & $\phantom{-}0.0\underline{8}180$ & $\phantom{-}0.0\underline{5}115$ & $\phantom{-}0.0\underline{6}979$ & $\phantom{-}0.0\underline{5}125$ \\
B & $.50$ & bias & $-0.0\underline{3}093$ & $\phantom{-}0.00\underline{3}04$ & $-0.0\underline{1}472$ & $-0.0\underline{1}457$ \\
& & s.e. & $\phantom{-}0.\underline{1}3225$ & $\phantom{-}0.0\underline{5}691$ & $\phantom{-}0.0\underline{9}936$ & $\phantom{-}0.0\underline{5}999$ \\
B & $.75$ & bias & $-0.\underline{1}4236$ & $\phantom{-}0.00\underline{6}68$ & $-0.0\underline{3}480$ & $-0.0\underline{1}519$ \\
& & s.e. & $\phantom{-}0.\underline{2}9333$ & $\phantom{-}0.0\underline{6}150$ & $\phantom{-}0.\underline{1}5309$ & $\phantom{-}0.0\underline{6}922$ \\
\midrule
\multicolumn{7}{c}{$N = 5000$} \\
\midrule
B & $.25$ & bias & $\phantom{-}0.000\underline{2}0$ & $\phantom{-}0.000\underline{4}7$ & $\phantom{-}0.000\underline{8}9$ & $-0.00\underline{8}72$ \\
& & s.e. & $\phantom{-}0.0\underline{2}620$ & $\phantom{-}0.0\underline{1}604$ & $\phantom{-}0.0\underline{2}153$ & $\phantom{-}0.0\underline{1}578$ \\
B & $.50$ & bias & $-0.00\underline{1}56$ & $\phantom{-}0.000\underline{5}4$ & $\phantom{-}0.000\underline{3}2$ & $-0.0\underline{1}636$ \\
& & s.e. & $\phantom{-}0.0\underline{4}268$ & $\phantom{-}0.0\underline{1}811$ & $\phantom{-}0.0\underline{3}081$ & $\phantom{-}0.0\underline{1}863$ \\
B & $.75$ & bias & $-0.0\underline{1}264$ & $\phantom{-}0.000\underline{6}2$ & $-0.00\underline{1}60$ & $-0.0\underline{2}564$ \\
& & s.e. & $\phantom{-}0.0\underline{9}704$ & $\phantom{-}0.0\underline{2}066$ & $\phantom{-}0.0\underline{4}753$ & $\phantom{-}0.0\underline{2}450$ \\
\bottomrule
\end{tabular}
  \end{table}
\end{solution}
\begin{mdframed}
  \begin{problem}
  \label{prob:2}
  Reproduzca la figura~1 del trabajo de Sadoon et al.
  \end{problem}
\end{mdframed}
\begin{mdframed}
  \begin{problem}
  \label{prob:3}
  Comente sobre el sesgo promedio de las estimaciones a medida que aumenta el valor del parámetro de persistencia $\rho$ para ambos estimadores: Arellano-Bond y SyS. ¿Varían las conclusiones al trabajar con muestras pequeñas o grandes?

  \end{problem}
\end{mdframed}
\printbibliography
\end{document}