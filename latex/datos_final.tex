%% !TeX root = filename
\documentclass[12pt,letterpaper,reqno,oneside]{amsart}
\usepackage{standalone}
\usepackage{subfiles}
\usepackage{subcaption}
\usepackage{dcolumn}
\usepackage[margin=1in]{geometry}
\usepackage[utf8]{inputenc}
\usepackage[T1]{fontenc}
% \usepackage[spanish,es-nodecimaldot,safe=none]{babel}
\usepackage{amsmath}
\usepackage{booktabs}
\usepackage{threeparttable} % for table notes
\usepackage[backend=biber,style=authoryear,natbib=true]{biblatex}
\addbibresource[label=main]{references.bib}
\usepackage{mdframed}
\usepackage{amsfonts}
\usepackage{physics}
% \usepackage{cmbright}
\usepackage{calc}
\usepackage{datetime}
\usepackage[style=american]{csquotes}
\usepackage{mathrsfs}
\usepackage{amssymb}
\usepackage{amsthm}
% \usepackage[extreme]{savetrees}
\usepackage{caption}
\usepackage{graphicx}
\usepackage{xfrac}
\usepackage{tikz}
\usepackage{pgfplots}
\usepackage{marginnote}
\usepackage{listings}
\usepackage[shortlabels]{enumitem}
\usepackage[hidelinks]{hyperref}
%\usepackage[useregional=text]{datetime2}
\usepackage{placeins}
\setcounter{tocdepth}{2}
\setcounter{secnumdepth}{2}
\pgfplotsset{compat=1.5}
% \setlength\parindent{0pt}


% Equations numbers are preceded by subsection
% \numberwithin{equation}{subsection}

% \theoremstyle{definition}
\newtheorem{thm}{Theorem}
\newtheorem{defn}{Definition}[subsection]
\newtheorem{prop}{Proposition}
\newtheorem{cor}{Corollary}[prop]
\newtheorem{remark}{Remark}
\newtheorem{lemma}{Lemma}[subsection]
% Define a new style for the 'problem' environment
\newtheoremstyle{problemstyle}
  {0pt} % Space above
  {\topsep} % Space below
  {} % Body font
  {0pt} % Indent amount
  {\bfseries} % Theorem head font
  {.} % Punctuation after theorem head
  {.5em} % Space after theorem head
  {} % Theorem head spec (can be left empty)

\theoremstyle{problemstyle} % Apply the new style for 'problem'
\newtheorem{problem}{Problem}
\newtheorem{appendixproblem}{Problem}
\theoremstyle{definition} % Revert to 'definition' style for subsequent environments
\newtheorem{solution}{Solution}[problem]
\newtheorem{appendixsolution}{Solution}[appendixproblem]
\renewcommand{\theappendixproblem}{\thesection}
\renewcommand{\theappendixsolution}{\theappendixproblem~(\alph{appendixsolution})}
\renewcommand{\thesolution}{\theproblem~(\alph{solution})}

% Section styling
% \renewcommand\thesection{\arabic{section}}
% \renewcommand\thesubsection{\thesection.\Alph{subsection}}
% \renewcommand\thesubsubsection{\thesubsection.\arabic{subsubsection}}

% Operators
\DeclareMathOperator{\ar}{AR}
\DeclareMathOperator{\ma}{MA}
\DeclareMathOperator{\arma}{ARMA}
\DeclareMathOperator{\garch}{GARCH}
\DeclareMathOperator{\plim}{p\!\lim}

% Packages for quarto


\begin{document}
\author[F. I. Tappata]{Felipe. I. Tappata}
% \address{Universidad Torcuato Di Tella, Buenos Aires, Argentina}
% \email{ftappata@mail.utdt.edu}
\newdate{date}{27}{06}{2025} % 2025-05-25
\date{\displaydate{date}}

\title[Final Exam]{Datos de Panel: Examen Final}
\begin{abstract}
  Este documento contiene la solución al examen final de la materia \emph{Datos de Panel} en la Universidad Torcuato Di Tella, primer trimestre de 2025.
  El trabajo consiste en la replicación de ciertas tablas y figuras del trabajo de \textcite{al-sadoonSimpleMethodsConsistent2019}, y un análisis de los resultados obtenidos.
  El código usado fue entregado junto a este documento, y se puede encontrar también en el repositorio \url{https://github.com/felipetappata/datos-final}.
  El software usado es Stata, con un uso auxiliar de Python y Bash para el procesamiento de \emph{output} y ayuda en la ejecución simultánea de simulaciones.
  El archivo \texttt{README.md} en la raíz del repositorio contiene instrucciones detalladas para la replicación de los resultados.
\end{abstract}
\maketitle
% Problem statements are enclosed in boxes, with solutions following.
% At the cost of slight repetition, I have attempted to make the solutions self-contained.
\begin{mdframed}
  \begin{problem}
  \label{prob:1}
  Reproduzca las tablas 1 a 3 del trabajo de Sadoon et al.
  \end{problem}
\end{mdframed}
\begin{solution}
  \begin{table}[htbp]
    \centering
    \caption{Sesgo promedio en el modelo $\ar(1)$ ($T = 7$, $500$~replicaciones)}
    \label{tab:table1}
    \begin{tabular}{@{}ccc*{4}{c}@{}}
\toprule
& & & \multicolumn{2}{c}{No endogenous} & \multicolumn{2}{c}{Endogenous} \\
& & & \multicolumn{2}{c}{selection} & \multicolumn{2}{c}{selection} \\
\cmidrule(lr){4-5} \cmidrule(lr){6-7}
Select. & & & (1) & (2) & (3) & (4) \\
Model & $\rho$ & & AB & SYS & AB & SYS \\
\midrule
\midrule
\multicolumn{7}{c}{$N = 500$} \\
\midrule
A & $.25$ & bias & $-0.00923$ & $0.00126$ & $-0.00725$ & $-0.00453$ \\
& & s.e. & $0.07124$ & $0.04541$ & $0.06283$ & $0.04565$ \\
A & $.50$ & bias & $-0.02277$ & $0.00189$ & $-0.01460$ & $-0.01062$ \\
& & s.e. & $0.11471$ & $0.05020$ & $0.09111$ & $0.05220$ \\
A & $.75$ & bias & $-0.10656$ & $0.00497$ & $-0.03942$ & $-0.01384$ \\
& & s.e. & $0.25931$ & $0.05628$ & $0.15219$ & $0.06312$ \\
\midrule
\multicolumn{7}{c}{$N = 5000$} \\
\midrule
A & $.25$ & bias & $-0.00169$ & $0.00035$ & $-0.00095$ & $-0.00552$ \\
& & s.e. & $0.02275$ & $0.01415$ & $0.02020$ & $0.01398$ \\
A & $.50$ & bias & $-0.00426$ & $0.00027$ & $-0.00238$ & $-0.01270$ \\
& & s.e. & $0.03678$ & $0.01595$ & $0.02951$ & $0.01620$ \\
A & $.75$ & bias & $-0.01483$ & $3.66 \times 10^{-6}$ & $-0.00567$ & $-0.02181$ \\
& & s.e. & $0.08148$ & $0.01827$ & $0.04828$ & $0.02084$ \\
\midrule
\multicolumn{7}{c}{$N = 500$} \\
\midrule
B & $.25$ & bias & $-0.01756$ & $0.00462$ & $0.04545$ & $0.00284$ \\
& & s.e. & $0.09246$ & $0.05977$ & $0.08114$ & $0.06054$ \\
B & $.50$ & bias & $-0.04786$ & $0.00560$ & $0.06469$ & $-0.00134$ \\
& & s.e. & $0.15303$ & $0.06639$ & $0.11690$ & $0.07018$ \\
B & $.75$ & bias & $-0.21836$ & $0.01353$ & $0.08699$ & $0.00616$ \\
& & s.e. & $0.34816$ & $0.07143$ & $0.18169$ & $0.07874$ \\
\midrule
\multicolumn{7}{c}{$N = 5000$} \\
\midrule
B & $.25$ & bias & $0.00113$ & $0.00041$ & $0.06046$ & $0.00130$ \\
& & s.e. & $0.03299$ & $0.01870$ & $0.02802$ & $0.01861$ \\
B & $.50$ & bias & $-0.00080$ & $0.00031$ & $0.09268$ & $-0.00392$ \\
& & s.e. & $0.05533$ & $0.02061$ & $0.04100$ & $0.02127$ \\
B & $.75$ & bias & $-0.01647$ & $0.00069$ & $0.15206$ & $-0.00861$ \\
& & s.e. & $0.12682$ & $0.02276$ & $0.06521$ & $0.02570$ \\
\bottomrule
\end{tabular}
  \end{table}
\end{solution}
\begin{mdframed}
  \begin{problem}
  \label{prob:2}
  Reproduzca la figura~1 del trabajo de Sadoon et al.
  \end{problem}
\end{mdframed}
\begin{mdframed}
  \begin{problem}
  \label{prob:3}
  Comente sobre el sesgo promedio de las estimaciones a medida que aumenta el valor del parámetro de persistencia $\rho$ para ambos estimadores: Arellano-Bond y SyS. ¿Varían las conclusiones al trabajar con muestras pequeñas o grandes?

  \end{problem}
\end{mdframed}
\printbibliography
\end{document}